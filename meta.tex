\chapter{Meta}
"Good communication is just as stimulating as black coffee, and just as hard to sleep after." -- Anne Morrow Lindbergh

"Our language is the reflection of ourselves. A language is an exact reflection of the character and growth of its speakers." -- Cesar Chavez
 
"You never change things by fighting the existing reality. To change something, build a new model that makes the existing model obsolete." --R. Buckminster Fuller\todo{pick one quote and format it correctly}

\section{Messaging your project}
The first thing you want to have is an elevator pitch. It should be short, like the length of an elevator ride. It’s a sentence you can say when asked, “What’s such-and-such?” or “What is your group about?” that will help people understand what you’re doing and evaluate why it might (or might not) be interesting to them. An elevator pitch is not the same as a slogan. It packs a lot of meaning into a short interaction.  

Many developers like knowing how things work -- this makes us special! It also means that the things that we find exciting about our projects may not be the things that others find exciting. When you’re speaking to the general public, focus on what your project does, not how it does those things. Even for potential new developers, you should still start with the what (and the why!) and *then* you can get into the how.  

Things that most users tend *not* to care about:\todo{make this a list}
What programming language you're using
What software license you're using
Whether you're using the latest and greatest technology or old tried and true tools
That your project or company won an award, especially if it's one they haven't heard of

Things users do tend to care about:\todo{make this a list}
What problem does it solve?
Is it easy to use?
Is it reliable? Flexible? Cost-effective?
Basically, is this software my best choice?

The basic structure of a good elevator pitch: “ABC is the best thing for certain kinds of people who want something awesome.“ The ingredients are:\todo{make this a list}
“ABC” - you’ve named the project
“the best” - find an adjective that sets your project apart from other similar things. It could be “the cheapest” or “the user-friendly” option. Maybe it’s “the most secure” or “the simplest” or “the biggest” -- whatever you pick it should be an adjective that distinguishes your project from others. 
“thing” - Whether it’s a database, a web framework, a programming language or a plugin, make sure you say what it is. If you are reinventing a certain type of tool, don’t use your made-up name unless it is very obvious to everyone who hears it what you mean. 
“for certain kinds of people” - Most things aren’t for everyone. Is your project for teachers or sysadmins? Maybe it’s for anyone who likes tracking their receipts or people who like to play short games on their phones. Figure out who is likely to use your software and then name them. 
“who want something awesome” - Once you figure out who your project is for, figure out what you are providing them. Superior organization? Fine-grained control? An interactive learning experience? Whatever it is, name it. 

Here are some examples for good elevator pitches:\todo{make this a list}
Moodle is the free web application that educators can use to create effective online learning sites.
Drupal is the easy web development platform that allows you to organize, manage and publish your content, with an endless variety of customization.
MediaGoblin is decentralized media hosting for people who want to share their stuff and keep control of it.

Your website is often the first place someone comes into contact with your project. Err on the side of being too oriented towards newcomers, rather than too little, while still providing clear paths for project regulars to get things done. Consider providing a few 101-type links for people who visit your site. If you use a lot of acronyms in your work (try not to use too many!), make sure the website’s front page spells them out. If there’s something that you want visitors to your site to do, include a call to action saying so. Make sure you say what your project is and who might use it. 

Your website can also tell visitors what kind of people are involved in your project. A picture can help tremendously, ideally of a group of happy-looking people who work on your project. If you post a calendar on the front page, then make sure it is someone’s job to make sure it’s always current. Nothing says “waste of time” more quickly than a calendar with a single event from two years ago.  

Do’s\todo{make this a list}
Include pictures of events
Describe what your project is for someone who has never heard of it
Keep your calendar up to date
Make sure links for more information, email addresses and IRC channels are current
Ask people to try your software, attend a meeting, sign up for the mailing list
Provide links to relevant background information, eg. “What is Free Software/Python/Ruby?” 

Don’ts\todo{make this a list}
Use lots of acronyms on the front page
Mention that the “regular meeting” is starting a half hour later without saying when and where the regular meeting regularly happens
Provide an email address that no one looks at
Make new visitors click through multiple pages to find out what your project is or find out where they can download your software

Ask for feedback on your materials and website whenever you make changes. Bribe friends with cookies or beer if you need to.

\section{Publicizing your project at a booth at a conference}

Tabling is when you have a table (sometimes called a booth) at a conference, festival or other large event. Look for events where people who are likely to be interested in your project are likely to be. Set a goal while casting a wide net. Consider having more than one level of goal-based activity at your table. Are you hiring? Fundraising? Growing your userbase? Publicizing your advocacy or user group?  Here are a few examples of potential goals and what you might have/do at your table to support that goal:\todo{make this a list}
We’re hiring -- have a sheet or card with current openings printed on it for taking and passing along, include the website for your job announcement page
We’re fundraising -- have a few levels of membership/support and a clear answer to the question, “How does my financial support help?”
We want more users -- have an install disk or a handout with the address of a web version for test driving. A computer running a simple demo is also helpful. 
We want more people in our group -- ask people to join the email list that announces your meetings
We’re doing advocacy -- ask people to sign a petition, or give out stickers that publicize the campaign. Ask for contact information so people can stay up to date with your work.  

You want every person to feel good about their interaction with your project, even if they can't help you with your specific goal at this time. They may be able to help you in the future or may talk you up to a friend or colleague who can help. 

How should the table look? Tidy, organized and fun. You always want to have at least two people at your table. One person at a table all alone looks forlorn and sad. Three people is ideal because everyone can take breaks without leaving just one person at the table. No one should be using their laptops at the table. The social hump of interrupting someone on a laptop makes you much less approachable. Plus, if you’re doing work in a noisy public space, then you’re doing a bad job of that work and a bad job of tabling. Always be standing and ready to engage. Lastly, make sure the table looks neat and has relevant information, like flyers or stickers with your URL on them that people can take home. 

Train your table staff so the interaction serves the goal. If you’re hiring, then tablers should ask people if they’re looking for work. If you’re looking for new contributors then ask people if they’re familiar with your project; then you can tell them what they don’t know and start to figure out if they are interested in your work. Make it a conversation by asking a question, listen to the answer and then respond in a way that lets the other person know you heard them. 

No one knows the answer to everything. Have a game plan for handling the questions that you can‘t answer. Maybe the question is too technical or is just another person’s area of expertise. Give people the IRC or the question-asking email and encourage them to get the answer there. Maybe they’re thinking of a different group or project. Don’t make things up or brush aside their comments. Do your best to be helpful and thank them for coming by your table. 

No matter what happens, keep it positive. Don't disparage people who were already at the table to the next batch of people at your table. Sometimes you'll encounter trolls. There are some polite ways you can get rid of them. 

Troll: “Isn’t my programming language of choice much better than yours because of some obscure weirdness that someone I respect from the internet said?”
Tabler: “Wow, you've really given this a lot of thought. Clearly, there's no way I could ever change your mind... Thanks for coming by!”\todo{format as dialog}

Then turn away. Seriously. Trolls are primarily looking for attention. Take it away and they’ll leave. 

Take breaks to eat or just decompress. For many people, it can be really overwhelming to be constantly "on." If you have enough tablers, schedule everyone in 4 hour shifts for tabling. 2.5 days in a row in a noisy exhibitor floor is a lot, and you will probably lose your voice before the end.

Tabling is hard work. Thank your tablers profusely and let them know how important their volunteering  is. Give them t-shirts if your project has them. If you had a numeric goal (30 new people on the email list), let your volunteers know whether you came close or exceeded the goal. Make sure you get feedback from your volunteers and incorporate it for future tables. Take pictures at the table and then blog about how awesome all your volunteers were. You’ll make this batch of volunteers feel appreciated and once you’ve shown how important and fun tabling is --  it might be even easier for you to find tablers next time!

\section{Places to spread the word}
There are many places to spread the word about your project and your outreach effort in particular - online and offline. In either case you need to ask yourself where the people you want to reach are likely to see your message. If that is not enough or not possible try to find the people who can reach the people you actually want to reach and go from there. Be creative! What are some groups you’re not reaching so far? Go for them! Some successful ones are for example events for women and their friends or events for designers.

Online: You can use for example your project’s mailing list, the news site of your project, the blog aggregator (often called a Planet), mailing lists for meetup groups or user’s groups related to your project, as well as various social media outlets your project might have on identi.ca, Twitter, Facebook, Diaspora, Google+ and others.

Offline: You can print flyers or little cards to give them away. Universities, schools, and hacker spaces could be a good place for them depending on the target audience of your event.

\section{Community metrics}
Doing outreach is demanding but also rewarding work. You want to track some metrics to make sure you and everyone helping you can see the effect and benefit of your work. What you are going to track will be specific to your project but here are a few suggestions:\todo{make this a list}
New contributors to your project
How many patches you receive
Participants in your events
Signups to your events
Retention rate from your events
Fixed bite-sized bugs
Mentoring requests
New mailing list members by outreach strategy

You want to track them monthly or whenever you have events.

Qualitative community surveys can also provide useful feedback. If you run an event, we recommend you send an exit survey to event participants to learn how to make the event better for next time and compare with similar events.
